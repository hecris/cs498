\documentclass{article}

\usepackage{amsmath}
\usepackage{amssymb}
\usepackage{amsthm}

\title{Blind Sort}
\author{Christopher He}
\date{\today}

\begin{document}
\maketitle
\section{Introduction}

We are given an array of integers to sort but we are not allowed to see the contents of the array.
The only operation we have is to swap any two elements. This swap operation will return if we
placed any of those two elements in its correct sorted position. Once an element is in its correct
sorted position, it will "freeze" and will no longer move. Originally, all elements are out of place.
We will explore a probabilistic algorithm
that sorts the array by randomly swapping elements and prove it is expected to sort in $\Theta(N^2)$ swaps.
We will also explore a deterministic algorithm to sort the array that also runs in $\Theta(N^2)$
time. Lastly, we will provide the model of an adversary that would force any algorithm to make
$\Omega(N^2)$ swaps to sort the array, thus establishing a lower bound.

In studying this problem, we will see that we are really working with derangements. A derangement is
a permutation of elements such that no element is in its correct position. For our problem, assume
a derangement of size N is an array A of integers \{1, 2, ... N\} such that $A[i] \neq i$ for all $i = 1, ..., N$.

We also introduce the concept of a good swap. A good swap is a swap that places at least one element in its correct
position. That is, a swap $(i, j)$ is good if $A[i] = j$ or $A[j] = i$. Call a good swap that freezes exactly one element
a $1$-swap and a good swap that freezes two elements a $2$-swap.

We familiarize the reader to the structures we are working with by considering a counting problem:
how many derangements of size $N$ are there with $K$ good swaps?

\subsection{Number of derangements with k good swaps}
Given a permutation, we can find the cycle decomposition of it. For example, given $A = [3,2,1,6,4,7,5]$
the cycle decomposition is $[3,1],[2],[6,7,5,4]]$. We first make some observations about the cycle
decomposition of a derangement. First, there would be no cycles of length 1, as this would represent
an element in its original position. Also, a cycle of length $2$ would represent a $2$-swap. Lastly, a cycle of length $l$ where
$l > 2$ would represent $l$ $1$-swaps. Given these observations, let $c_2$ be the number of cycles of length $2$ ($2$-cycle)
in a derangement,
then the number of good swaps is $N - c_2$.

If we find the expected number of good swaps, this could also give us insight into the expected number of $2$-cycles.
Note we could've found the expected number of $2$-cycles directly, but we saw the relation after the fact.

Let s be the number of good swaps in a derangement D of size N. We are trying to find
\[ E[s] = \sum_{k = \lfloor N/2 \rfloor}^{N} p_k \times k \]

Where $p_k$ is the probability of a derangement having $k$ good swaps. We can define this as

\[ \frac{\text{number of derangements of size $N$ with $k$ good swaps}}{\text{total number of derangements of size $N$}} \]

It can be shown that the number of derangements of size N is exactly $\lfloor \frac{N! + 1}{e} \rfloor$.
But how many derangements of size $N$ have $k$ good swaps? We propose a recurrence to answer this question.

We start with a naive recurrence:
Let T(N, k) be the number of derangements of size $N$ with $k$ good swaps where
$k <= N$.
We define a recurrence for T(N, k) based on the following idea:

Pick $i$ elements from the $N$ elements, where $i > 1$. With these $i$ elements, we will make a cycle of length $i$.

If we make a cycle of length $2$, that contributes $1$ good swap.
We would like the rest of the $N - 2$ vertices to contribute $k - 1$
good swaps. Assume we have $T(N - 2, k - 1)$

If we make a cycle of length $i$ where $i > 2$, that contributes
$i$ good swaps. Assume we have $T(N - i, k - i)$.

Let $C_i$ be the number of cycles we can make of length $i$.
There are ${N \choose i}$ ways to pick $i$ vertices. With these $i$ vertices,
there are $(i-1)!$ cycles we can make (fix a starting element, there
are $(i - 1)$ choices for the second vertex in the cycle,
then $(i - 2)$, and so on). 

So $C_i = {N \choose i} \times (i - 1)!$.

Putting this all together, we have the following recurrence:
\[ T(N, k) =
C_2 \times T(N - 2, k - 1) +
\sum_{i = 2}^{k}{C_i \times T(N-i, k-i)} \]

This recurrence is naive because of the following observation:
Take for example a derangement of 11 elements. Let's say we split this
derangements into cycles of lengths 4, 3, 2, and 2. We counted this
arrangement $4!$ times since there are that many ways to permute these
cycles. Thus we define another recurrence $T^\prime(N, k, s)$, the number
of ways to make a derangement of N vertices with k good swaps
with s cycles. The naive reccurence is slightly modified:

\[ T^\prime(N, k, s) =
C_2 \times T^\prime(N - 2, k - 1, s-1) +
\sum_{i = 2}^{k}{C_i \times T^\prime(N-i, k-i, s-1)} \]
Thus our final solution is
\[ T(N, K) = \sum_{s = 1}^{N}{\frac{T^\prime(N, K, s)}{s!}} \]

Running this algorithm shows that $E[S] = N - \frac{1}{2}$, which reveals that $E[c_2] = \frac{1}{2}$.

\section{Deterministic Algorithm}
We propose a simple deterministic algorithm: starting at index $i = 0$, we repeatedly swap the element
at index $i$ with all indices $j > i$ in a linear fashion, until we freeze index $i$ in which case we go onto the 
next iteration of $i$. It is easy to see that this runs in $O(N^2)$ time. In fact, for any $N$ we can construct an input
that takes $\Theta(N^2)$ swaps. Let $N = 8$, for example, and consider the input $[5, 6, 7, 8, 4, 3, 2, 1]$.
We can characterize this as putting the half of biggest elements in the beginning of the array, and the half of smallest elements at
the end of the array in reverse order. In essence, we are trying to maximize the number of swaps to place the smallest
elements where they belong.
The first iteration will take $N - 1 = 7$ swaps until $1$ is placed in its correct position. Observe that all elements
between $5$ and $1$ are circularly shifted right by $1$ in the process. So we have

$[5, 6, 7, 8, 4, 3, 2, 1] \rightarrow 7$ swaps

$[1, 5, 6, 7, 8, 4, 3, 2] \rightarrow 6$ swaps

$[1, 2, 5, 6, 7, 8, 4, 3] \rightarrow 5$ swaps

$[1, 2, 3, 5, 6, 7, 8, 4] \rightarrow 4$ swaps

$[1, 2, 3, 4, 5, 6, 7, 8]$

And the array is sorted. In general, each iteration $i$ performs $N - i$ swaps.
For an even $N$, the number of iterations is $N / 2$. So, the total number of swaps is
$(N - 1) + ... + \frac{N}{2}$. We can describe this as a difference of two sums and solve:

\begin{align}
    & \sum_{i = 1}^{N - 1}{i} - \sum_{i = 1}^{\frac{N}{2} - 1}{i}  \\\nonumber
    &= \frac{N(N-1)}{2} - \frac{(\frac{N}{2})(\frac{N}{2} - 1)}{2} \\\nonumber
    &= \frac{N(N-1)}{2} - \frac{(2N)(\frac{N}{2} - 1)}{8} \\\nonumber
    &= \frac{N^2 - N}{2} - \frac{N^2 - 2N}{8} \\\nonumber
    &= \frac{4N^2 - 4N}{8} - \frac{N^2 - 2N}{8} \\\nonumber
    &= \frac{3N^2 - 2N}{8} = \frac{N(3N - 2)}{8} \\\nonumber
\end{align}

This attempt at a worst case input will be seen again when we describe the adversary for any algorithm.
It turns out that when the deterministic algorithm "plays" the adversary, the adversary will force
the algorithm to take this exact amount of steps (and claims this was the input all along).

It is interesting to note that the worst case does not always follow this structure. For $N = 10$
there is another derangement that make the deterministic algorithm take $1$ more swap than for
the derangement we constructed, namely $[6, 7, 8, 9, 3, 4, 5, 2, 1, 0]$.
We spent some time trying to find a way to generalize this structure. One can revisit this by looking
at the cycle decomposition of this worst case derangement.

Before, we observed that each iteration of the algorithm would circularly shift a subarray of elements.
Because of this, some elements can be unintentionally frozen. This led to some roadblocks
in finding a nice recurrence for the running time of the deterministic algorithm. Although, because the
change to the array can be characterized nicely (circular shift to some elements), we did spend some time
trying to find how many derangements are there such that 
$k$ elements are in the index before their correct ones (so $k$ elements would be shifted and frozen).
If we found a nice answer to this, we could've
described the running time as 

$ T(N) = \sum_{k = 1}^{N}{p_k * T(N - k)}$

Although we were not able to find a recurrence for the running time, through simulations we found that
the deterministic algorithm took around $N^2/6$ swaps.

\subsection{Case where array has k distinct values}
As the introduction mentioned, we have a lower bound of $\Omega(N^2)$ swaps on any blind sorting algorithm.
In this section,
we go over a case that would break our lower bound. In particular, if the array of length $N$ only
contains numbers from a set of $K$ elements (for example $\{1, ..., K\}$) for $ 1 \leq K \leq N $, then we have a 
deterministic $O(NK)$ algorithm as follows: we always maintain a pointer to the first nonfrozen index
from the left, call it $i$. In each iteration, we swap $i$ with all indices $j$ such that $j > i$. 
At any time if index $i$ is frozen, we increment $i$ by 1. In each iteration, when we finish scanning all $j$, we
repeat and our algorithm stops when $i > N$, that is, all elements are frozen.
Notice that in each iteration $i$, all instances of the $i$th smallest element in $K$ will be fixed.
Thus, we will have at most $K$ iterations, each performing $O(N)$ swaps so in total $O(NK)$ swaps.
Note that for the general case $K = N$, this algorithm is still $O(N^2)$.

\section{Probabilistic Algorithm}
In the previous section we describe a deterministic algorithm to sort the array. We introduce
a simple randomized algorithm: repeatedly swap random pairs of elements until the array is sorted
(or all elements are frozen). It turns out that this algorithm runs in expected $\Theta(N^2)$
time. 

\subsection{n/2 and n/8 bounds}
As mentioned before, the number of good swaps in a derangement size $N$ 
is $N - c_2$ where $c_2$ is the number
of $2$-cycles. Since $c_2$ must be between $0$ and $N/2$, then the number of good swaps
is between $N$ and $N/2$. We consider the best case where there are $N$ good swaps.
Then the probability of a good swap is $\frac{N}{{N \choose 2}} = \frac{2}{N - 1}$.
Given the probability of a good swap, we expect to perform $\frac{N - 1}{2}$ swaps before we 
freeze at least one element. Since we are considering the best case, let's assume our
good swap freezes two elements. Then we now have derangement of size $N - 2$ that, in the best
case, has $N - 2$ good swaps. Thus we have the sum
$\frac{1}{2}((N - 1) + (N - 3) + (N - 5) + ... + 3 + 1) = \frac{1}{2}(\frac{N^2}{4}) = \frac{N^2}{8} $.

Similarly, let's consider the worst case where there are $N/2$ good swaps. Then the probability
of a good swap is $\frac{1}{N - 1}$ so we expect to perform $N - 1$ swaps before freezing
something. Since we are considering the worst case, let's assume we only freeze one element.
So now we have a derangement of size $N - 1$, and we expect to perform $N - 2$ swaps before
freezing. Thus we have the sum
$ N - 1 + N - 2 + ... = {N \choose 2} $

Thus the expected running time of the random swapping algorithm is $\Theta(N^2)$.
Although we have these worst and best case lower bounds $\frac{N^2}{2}$ and $\frac{N^2}{8}$,
through simulations we observed that it takes almost exactly $\frac{N^2}{4}$ swaps.
In the next sections, we derive a recurrence to get a more exact expected running time.

\subsection{n/4 bound - show derivation of p1 and p2 but explain why recurrence is naive}
introduce the recurrence, show how p1 and p2 are derived, but state that the uniformity
does not hold when recursing, but still through simulation the running time seems very close to $n^2/4$
\subsection{swaps within derangements of size n lead to stable distribution - degree argument}
here we attempt to explain why although derangements don't behave uniformly
like permutations, maybe we can assume some sort of uniformity. 
show how we find that the
degree, also known as the number of 0 swaps in a derangement, is ${n \choose 2} - n + l_2 $, 
we see the degree is dominated by the ${n \choose 2}$ term
\subsection{other mentions?}
the degree from n - 1 derangements to n derangements is uniform: n * (n - 1)
the degree from n - 2 derangements to n derangements is uniform: n * (n - 1) / 2

\section{Adversary}
\subsection{describing the adversary}
setting up the bipartite graph, maintaining a matching, explain that all elements
are determined when a matching is unique.
\subsection{implementation}
talk about or show code of the implementation
\subsection{$n^2$ lower bound}
given the setup of the adversary, show that $\Omega(n^2)$ edges must have been removed 
before reaching a unique matching

\end{document}
