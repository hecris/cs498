\title{Algorithm to find number of derangements of length N with K good swaps}
\author{
    Christopher He
}
% \date{\today}

\documentclass[12pt]{article}
\usepackage{amsmath}

\begin{document}
\maketitle

% \begin{abstract}
% This is the paper's abstract \ldots
% \end{abstract}

\section{Introduction}

We consider a derangement of size $N$ to be an array $D$ that is a permutation of ${0, …, N - 1}$
where no element is in its correct sorted position.
That is, for all $i, 0 <= i < N, D[i] \neq i$. Note that in our definition of derangement,
all elements are distinct.

% \paragraph{Outline}
% The remainder of this article is organized as follows.
% Section~\ref{previous work} gives account of previous work.
% Our new and exciting results are described in Section~\ref{results}.
% Finally, Section~\ref{conclusions} gives the conclusions.

A good swap in a derangement $D$ of size $N$ is a pair $(i, j), 0 <= i < j < N$
such that $D[i] = j$ or $D[j] = i$.
% In English, a swap is good if it places at least one element in its correct sorted position.

Let s be the number of good swaps in a derangement D of size N.
First observe that for every element $x$ in $D$, there is at most one good swap that places $x$ in its
correct position, so $s <= N$. Also, $s > \lfloor N / 2 \rfloor$, this follows from the fact that a good swap,
 in the best case, places two elements in their correct positions (i.e $D[i] = j$ AND $D[j] = i$).

So we know that $ \lfloor N/2 \rfloor < s <= N$. using these bounds, we can show that the expected number of random
swaps to sort D is between $N^2/8$ and $N^2/2$. To get a more exact running time, it could be
helpful to find the expected number of good swaps. This is defined as

\[ E[s] = \sum_{k = \lfloor N/2 \rfloor}^{N} p_k \times k \]

Where $p_k$ is the probability of a derangement having $k$ good swaps. We can define this as

\[ \frac{\text{number of derangements with $k$ good swaps}}{\text{total number of derangements}} \]

It can be shown that the number of derangements of size N is exactly $\lfloor \frac{N! + 1}{e} \rfloor$.
But how many derangements of size $N$ have $K$ good swaps?


\section{Model as Graph}
Given a derangement $D$ of size $N$ we construct a directed graph $G(V, E)$ where $V = \{0, ..., N - 1\}$
and $(i, j) \in E$ if $D[i] = j$. First observe that there are no self-loops in this graph because in 
a derangement there is no element in its correct position. Secondly, observe that each vertex $i$ has an outdegree
of exactly $1$, that is, $D[i] = j$ for some $j$ and if $D[i] = j$ and $D[i] = k$ then $j = k$. A similar argument shows that each
vertex has an indegree of $1$. With these observations, the graph must be a connected cycle graph, or if
disconnected, decomposed into multiple cycle graphs each with 2 or more nodes.
Let's call a graph with these properties a \textbf{derangement graph}.
The number of good swaps in $D$ is almost the number of edges in its derangement graph, except if we have a cycle of $2$ nodes,
that represents $1$ good swap, whereas if we have a cycle of $m$ nodes where $m > 2$ then that cycle contributes
$m$ good swaps ($m$ nodes implies $m$ edges, each being an instance where $D[i] = j$ so one of $(i, j)$ or 
$(j, i)$ is a good swap).

We then rephrase our original problem to the following: how many derangement graphs with
vertices $\{0, ..., N - 1\}$ are there with $k$ good swaps?

\section{Recurrence}
We start with a naive recurrence:
Let T(N, k) be the number of derangement graphs with vertices $\{0, ..., N - 1\}$ with $k$ good swaps where
$k <= N$.
We define a recurrence for T(N, K) based on the following idea:

Pick $i$ vertices from the $N$ vertices. With these $i$ vertices, we will make a cycle of length $i$, so $i$ must be greater than 1.

If we make a cycle of length $2$, that contributes $1$ good swap.
We would like the rest of the $n - 2$ vertices to contribute $k - 1$
good swaps. Assume we have $T(n - 2, k - 1)$

If we make a cycle of length $i$ where $i > 2$, that contributes
$i$ good swaps. Assume we have $T(n - i, k - i)$.

Let $C_i$ be the number of cycles we can make of length $i$.
There are ${i \choose 2}$ ways to pick $i$ vertices. With these $i$ vertices,
there are $i!$ cycles we can make (fix a starting vertex, there
are $(i - 1)$ choices for the second vertex in the cycle,
then $(i - 2)$, and so on). 

So $C_i = (i - 1)! * {i \choose 2}$.

Putting this all together, we have the following recurrence:
\[ T(N, k) =
C_2 \times T(N - 2, k - 1) +
\sum_{i = 2}^{k}{C_i \times T(N-i, k-i)} \]

This recurrence is naive because of the following observation:
Take for example a graph of 8 vertices. Let's say we split this
graph into cycles of 4, 3, 2, and 1. We counted this
arrangement $4!$ times since there are that many ways to permute these
cycles. Thus we define another recurrence $T^\prime(N, K, S)$, the number
of ways to make a derangement graph of N vertices with K good swaps
with S connected components. The naive reccurence is slightly modified:

\[ T^\prime(N, k, s) =
C_2 \times T^\prime(N - 2, k - 1, s-1) +
\sum_{i = 2}^{k}{C_i \times T^\prime(N-i, k-i, s-1)} \]

Thus our final solution is

$ T(N, K) = \sum_{s = 1}^{N - 1}{\frac{T^\prime(N, K, s)}{s!}} $

\section{Conclusions}\label{conclusions}
Running this algorithm shows that the expected number of good swaps
in a derangement of size $N$ is almost exactly $N - 0.5$.

% \bibliographystyle{abbrv}
% \bibliography{main}

\end{document}
This is never printed

