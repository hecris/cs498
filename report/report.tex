\documentclass{article}

\usepackage{minted}
\usepackage{amsmath}
\usepackage{amssymb}
\usepackage{amsthm}

\title{Blind Sort}
\author{Christopher He}
\date{\today}

\begin{document}
\maketitle
\section{Introduction}

We are given an array of integers to sort but we are not allowed to see the contents of the array.
The only operation we have is to swap any two elements. This swap operation will return if we
placed any of those two elements in its correct sorted position. Once an element is in its correct
sorted position, it will "freeze" and will no longer move. Originally, all elements are out of place.
We will explore a probabilistic algorithm
that sorts the array by randomly swapping elements and prove it is expected to sort in $\Theta(N^2)$ swaps.
We will also explore a deterministic algorithm to sort the array that also runs in $\Theta(N^2)$
time. Lastly, we will provide the model of an adversary that would force any algorithm to make
$\Omega(N^2)$ swaps to sort the array, thus establishing a lower bound.

In studying this problem, we will see that we are really working with derangements. A derangement is
a permutation of elements such that no element is in its correct position. For our problem, assume
a derangement of size N is an array A of integers \{1, 2, ... N\} such that $A[i] \neq i$ for all $i = 1, ..., N$.

We also introduce the concept of a good swap. A good swap is a swap that places at least one element in its correct
position. That is, a swap $(i, j)$ is good if $A[i] = j$ or $A[j] = i$. Call a good swap that freezes exactly one element
a $1$-swap and a good swap that freezes two elements a $2$-swap.

We familiarize the reader to the structures we are working with by considering a counting problem:
how many derangements of size $N$ are there with $K$ good swaps?

\subsection{Number of derangements with k good swaps}
Given a permutation, we can find the cycle decomposition of it. For example, given $A = [3,2,1,6,4,7,5]$
the cycle decomposition is $[3,1],[2],[6,7,5,4]]$. We first make some observations about the cycle
decomposition of a derangement. First, there would be no cycles of length 1, as this would represent
an element in its original position. Also, a cycle of length $2$ would represent a $2$-swap. Lastly, a cycle of length $l$ where
$l > 2$ would represent $l$ $1$-swaps. Given these observations, let $c_2$ be the number of cycles of length $2$ ($2$-cycle)
in a derangement,
then the number of good swaps is $N - c_2$.

If we find the expected number of good swaps, this could also give us insight into the expected number of $2$-cycles.
Note we could've found the expected number of $2$-cycles directly, but we saw the relation after the fact.

Let s be the number of good swaps in a derangement D of size N. We are trying to find
\[ E[s] = \sum_{k = \lfloor N/2 \rfloor}^{N} p_k \times k \]

Where $p_k$ is the probability of a derangement having $k$ good swaps. We can define this as

\[ \frac{\text{number of derangements of size $N$ with $k$ good swaps}}{\text{total number of derangements of size $N$}} \]

It can be shown that the number of derangements of size N is exactly $\lfloor \frac{N! + 1}{e} \rfloor$.
But how many derangements of size $N$ have $k$ good swaps? We propose a recurrence to answer this question.

We start with a naive recurrence:
Let T(N, k) be the number of derangements of size $N$ with $k$ good swaps where
$k <= N$.
We define a recurrence for T(N, k) based on the following idea:

Pick $i$ elements from the $N$ elements, where $i > 1$. With these $i$ elements, we will make a cycle of length $i$.

If we make a cycle of length $2$, that contributes $1$ good swap.
We would like the rest of the $N - 2$ vertices to contribute $k - 1$
good swaps. Assume we have $T(N - 2, k - 1)$

If we make a cycle of length $i$ where $i > 2$, that contributes
$i$ good swaps. Assume we have $T(N - i, k - i)$.

Let $C_i$ be the number of cycles we can make of length $i$.
There are ${N \choose i}$ ways to pick $i$ vertices. With these $i$ vertices,
there are $(i-1)!$ cycles we can make (fix a starting element, there
are $(i - 1)$ choices for the second vertex in the cycle,
then $(i - 2)$, and so on). 

So $C_i = {N \choose i} \times (i - 1)!$.

Putting this all together, we have the following recurrence:
\[ T(N, k) =
C_2 \times T(N - 2, k - 1) +
\sum_{i = 2}^{k}{C_i \times T(N-i, k-i)} \]

This recurrence is naive because of the following observation:
Take for example a derangement of 11 elements. Let's say we split this
derangements into cycles of lengths 4, 3, 2, and 2. We counted this
arrangement $4!$ times since there are that many ways to permute these
cycles. Thus we define another recurrence $T^\prime(N, k, s)$, the number
of ways to make a derangement of N vertices with k good swaps
with s cycles. The naive reccurence is slightly modified:

\[ T^\prime(N, k, s) =
C_2 \times T^\prime(N - 2, k - 1, s-1) +
\sum_{i = 2}^{k}{C_i \times T^\prime(N-i, k-i, s-1)} \]
Thus our final solution is
\[ T(N, K) = \sum_{s = 1}^{N}{\frac{T^\prime(N, K, s)}{s!}} \]

Running this algorithm shows that $E[S] = N - \frac{1}{2}$, which reveals that $E[c_2] = \frac{1}{2}$.

\section{Deterministic Algorithm}
We propose a simple deterministic algorithm: starting at index $i = 0$, we repeatedly swap the element
at index $i$ with all indices $j > i$ in a linear fashion, until we freeze index $i$ in which case we go onto the 
next iteration of $i$. It is easy to see that this runs in $O(N^2)$ time. In fact, for any $N$ we can construct an input
that takes $\Theta(N^2)$ swaps. Let $N = 8$, for example, and consider the input $[5, 6, 7, 8, 4, 3, 2, 1]$.
We can characterize this as putting the half of biggest elements in the beginning of the array, and the half of smallest elements at
the end of the array in reverse order. In essence, we are trying to maximize the number of swaps to place the smallest
elements where they belong.
The first iteration will take $N - 1 = 7$ swaps until $1$ is placed in its correct position. Observe that all elements
between $5$ and $1$ are circularly shifted right by $1$ in the process. So we have

$[5, 6, 7, 8, 4, 3, 2, 1] \rightarrow 7$ swaps

$[1, 5, 6, 7, 8, 4, 3, 2] \rightarrow 6$ swaps

$[1, 2, 5, 6, 7, 8, 4, 3] \rightarrow 5$ swaps

$[1, 2, 3, 5, 6, 7, 8, 4] \rightarrow 4$ swaps

$[1, 2, 3, 4, 5, 6, 7, 8]$

And the array is sorted. In general, each iteration $i$ performs $N - i$ swaps.
For an even $N$, the number of iterations is $N / 2$. So, the total number of swaps is
$(N - 1) + ... + \frac{N}{2}$. We can describe this as a difference of two sums and solve:

\begin{align}
    & \sum_{i = 1}^{N - 1}{i} - \sum_{i = 1}^{\frac{N}{2} - 1}{i}  \\\nonumber
    &= \frac{N(N-1)}{2} - \frac{(\frac{N}{2})(\frac{N}{2} - 1)}{2} \\\nonumber
    &= \frac{N(N-1)}{2} - \frac{(2N)(\frac{N}{2} - 1)}{8} \\\nonumber
    &= \frac{N^2 - N}{2} - \frac{N^2 - 2N}{8} \\\nonumber
    &= \frac{4N^2 - 4N}{8} - \frac{N^2 - 2N}{8} \\\nonumber
    &= \frac{3N^2 - 2N}{8} = \frac{N(3N - 2)}{8} \\\nonumber
\end{align}

This attempt at a worst case input will be seen again when we describe the adversary for any algorithm.
It turns out that when the deterministic algorithm "plays" the adversary, the adversary will force
the algorithm to take this exact amount of steps (and claims this was the input all along).

It is interesting to note that the worst case does not always follow this structure. For $N = 10$
there is another derangement that make the deterministic algorithm take $1$ more swap than for
the derangement we constructed, namely $[6, 7, 8, 9, 3, 4, 5, 2, 1, 0]$.
We spent some time trying to find a way to generalize this structure. One can revisit this by looking
at the cycle decomposition of this worst case derangement.

Before, we observed that each iteration of the algorithm would circularly shift a subarray of elements.
Because of this, some elements can be unintentionally frozen. This led to some roadblocks
in finding a nice recurrence for the running time of the deterministic algorithm. Although, because the
change to the array can be characterized nicely (circular shift to some elements), we did spend some time
trying to find how many derangements are there such that 
$k$ elements are in the index before their correct ones (so $k$ elements would be shifted and frozen).
If we found a nice answer to this, we could've
described the running time as 

$ T(N) = \sum_{k = 1}^{N}{p_k * T(N - k)}$

Although we were not able to find a recurrence for the running time, through simulations we found that
the deterministic algorithm took around $N^2/6$ swaps.

\subsection{Case where array has k distinct values}
As the introduction mentioned, we have a lower bound of $\Omega(N^2)$ swaps on any blind sorting algorithm.
In this section,
we go over a case that would break our lower bound. In particular, if the array of length $N$ only
contains numbers from a set of $K$ elements (for example $\{1, ..., K\}$) for $ 1 \leq K \leq N $, then we have a 
deterministic $O(NK)$ algorithm as follows: we always maintain a pointer to the first nonfrozen index
from the left, call it $i$. In each iteration, we swap $i$ with all indices $j$ such that $j > i$. 
At any time if index $i$ is frozen, we increment $i$ by 1. In each iteration, when we finish scanning all $j$, we
repeat and our algorithm stops when $i > N$, that is, all elements are frozen.
Notice that in each iteration $i$, all instances of the $i$th smallest element in $K$ will be fixed.
Thus, we will have at most $K$ iterations, each performing $O(N)$ swaps so in total $O(NK)$ swaps.
Note that for the general case $K = N$, this algorithm is still $O(N^2)$.

\section{Probabilistic Algorithm}
In the previous section we describe a deterministic algorithm to sort the array. We introduce
a simple randomized algorithm: repeatedly swap random pairs of elements until the array is sorted
(or all elements are frozen). It turns out that this algorithm runs in expected $\Theta(N^2)$
time. 

\subsection{Expected Number of Swaps is Between $N^2/2$ and $N^2/8$}
As mentioned before, the number of good swaps in a derangement size $N$ 
is $N - c_2$ where $c_2$ is the number
of $2$-cycles. Since $c_2$ must be between $0$ and $N/2$, then the number of good swaps
is between $N$ and $N/2$. We consider the best case where there are $N$ good swaps.
Then the probability of a good swap is $\frac{N}{{N \choose 2}} = \frac{2}{N - 1}$.
Given the probability of a good swap, we expect to perform $\frac{N - 1}{2}$ swaps before we 
freeze at least one element. Since we are considering the best case, let's assume our
good swap freezes two elements. Then we now have derangement of size $N - 2$ that, in the best
case, has $N - 2$ good swaps. Thus we have the sum
$\frac{1}{2}((N - 1) + (N - 3) + (N - 5) + ... + 3 + 1) = \frac{1}{2}(\frac{N^2}{4}) = \frac{N^2}{8} $.

Similarly, let's consider the worst case where there are $N/2$ good swaps. Then the probability
of a good swap is $\frac{1}{N - 1}$ so we expect to perform $N - 1$ swaps before freezing
something. Since we are considering the worst case, let's assume we only freeze one element.
So now we have a derangement of size $N - 1$, and we expect to perform $N - 2$ swaps before
freezing. Thus we have the sum
$ N - 1 + N - 2 + ... = {N \choose 2} $

Thus the expected running time of the random swapping algorithm is $\Theta(N^2)$.
Although we have these worst and best case lower bounds $\frac{N^2}{2}$ and $\frac{N^2}{8}$,
through simulations we observed that it takes almost exactly $\frac{N^2}{4}$ swaps.
In the next sections, we derive a recurrence to get a more exact expected running time.

\subsection{Recurrence for a more exact running time $N^2/4$}

Observe that performing a swap freezes 0, 1, or 2 elements. Thus we can write a recurrence
for the expected number of swaps to sort an derangement of size N.
\begin{align}
    & T(N) = p_0T(N) + p_1T(N - 1) + p_2T(N - 2) + 1 \\\nonumber
    & T(N) - p_0T(N) = p_1T(N - 1) + p_2T(N - 2) + 1 \\\nonumber
    & T(N)(1 - p_0) = p_1T(N - 1) + p_2T(N - 2) + 1 \\\nonumber
    & T(N) = \frac{p_1T(N - 1) + p_2T(N - 2) + 1}{1 - p_0} \\\nonumber
    & T(N) = \frac{p_1T(N - 1) + p_2T(N - 2) + 1}{p1 + p2} \\\nonumber
\end{align}

Where $p_1$ and $p_2$ are the probabilities of a $1$-swap and a $2$-swap, respectively.
So how can we find $p_1$ and $p_2$?
Let's start by looking for $p_2$. Let's say we have
a derangement A of size N and we pick $i$ and $j$. If $(i ,j)$ is a $2$-swap, that is,
$A[i] = j$ and $A[j] = i$. We know the position of two elements and the rest of the $N - 2$ elements
are deranged. So $p_2 = \frac{!(N - 2)}{!N}$. Since the number of derangements of size N is approximately
$\frac{N!}{e}$, then $p_2 \approx \frac{1}{N(N - 1)}$.
Now we try to find $p_1$. Let $g$ be the number of good swaps, $g_1$ be the number of $1$-swaps
and $g_2$ be the number of $2$-swaps. We have
\begin{align}
    & E[g] = E[g_1] + E[g_2] \\\nonumber
    & N - \frac{1}{2} = E[g_1] + \frac{1}{2} \\\nonumber
    & E[g_1] = N - 1  \\\nonumber
\end{align}
We have
\begin{align}
    p_1 * {N \choose 2} = E[g_1] \\\nonumber
    p_1 = E[g_1] * \frac{2}{N(N - 1)} \\\nonumber
    p_1 = (N - 1) * \frac{2}{N(N - 1)} \\\nonumber
    p_1 = \frac{2}{N}
\end{align}

Plugging in these values for $p_1$ and $p_2$ in the recurrence and running it shows
that $T(N) \approx \frac{N^2}{4}$. We know use the substitution method to prove that
$\lim_{N \to \infty} \frac{T(N)}{N^2} = \frac{1}{4}$.

We first want to show that $T(n) > cn^2$ where $c > \frac{1}{4}$. Assume $T(m) > cm^2$ for $m < n$.
Then we have
\begin{align}
    & T(n) = \frac{p_1T(n - 1) + p_2T(n - 2) + 1}{p_1 + p_2} \\\nonumber
    &  > \frac{c(p_1(n - 1)^2 + p_2(n - 2)^2) + 1}{p_1 + p_2} \\\nonumber
    &=  {(c(\frac{2}{n}(n - 1)^2 + \frac{1}{n(n-1)}(n - 2)^2) + 1)}{\frac{n(n - 1)}{2n - 1}} \\\nonumber
    &=  \frac{c(2(n - 1)^3 + (n - 2)^2) + n(n-1)}{2n - 1} \\\nonumber
    &=  \frac{c(2n^3 - 5n^2 + 2n + 2) + n(n-1)}{2n - 1} \\\nonumber
    &=  \frac{c(2n^3 - n^2) + c(-4n^2 + 2n + 2) + n(n-1)}{2n - 1} \\\nonumber
    &=  cn^2 - \frac{c(4n^2 - 2n - 2) - n(n-1)}{2n - 1} \\\nonumber
    &=  cn^2 - \frac{n^2(4c - 1) - 2nc + n - 2c}{2n - 1} \\\nonumber
\end{align}
We are interested in showing $T(n) > cn^2$ so we want $4c - 1 < 0$. this means $c > 1/4$.
Thus we have a lower bound: $T(n) > \frac{1}{4}n^2$.

We now use the substitution method again to show $T(n) < \frac{1}{4}n(n + 1)$. Assume
$T(m) < \frac{1}{4}m(m+1)$ for $m < n$. Then we have
\begin{align}
    & T(n) = \frac{p_1T(n - 1) + p_2T(n - 2) + 1}{p_1 + p_2} \\\nonumber
    &  < \frac{c(p_1n(n - 1) + p_2(n - 2)(n - 1)) + 1}{p_1 + p_2} \\\nonumber
    &=  {(c(\frac{2}{n}n(n - 1) + \frac{1}{n(n-1)}(n - 2)(n-1)) + 1)}{\frac{n(n - 1)}{2n - 1}} \\\nonumber
    &=  \frac{c(2n(n - 1)^2 + (n - 2)(n - 1)) + n(n-1)}{2n - 1} \\\nonumber
    &=  \frac{c(2n^3 - 3n^2 - n + 2) + n(n-1)}{2n - 1} \\\nonumber
    &=  \frac{c(2n^3 + n^2 - n) + c(-4n^2 + 2) + n(n-1)}{2n - 1} \\\nonumber
    &=  cn(n + 1) - \frac{c(4n^2 - 2) - n(n-1)}{2n - 1} \\\nonumber
    &=  cn(n +1) - \frac{n^2(4c - 1) - 2c - n}{2n - 1} \\\nonumber
\end{align}
We are interested in showing $T(n) < cn(n + 1)$ so we want $4c - 1 \geq 0$. this means $c \leq 1/4$.
Thus we have a upper bound: $T(n) \leq \frac{1}{4}n(n + 1)$.

We have the following result:
\begin{align}
    &  \frac{1}{4}n^2 < T(n) \leq \frac{1}{4}n(n + 1)
\end{align}
Thus $\lim_{N \to \infty} \frac{T(N)}{N^2} = \frac{1}{4}$.

Now observe that to calculate $p_1$ and $p_2$ we assumed we started with any random derangement
with equal probability. If we, for example, perform a $1-swap$, it is expected that any 
derangement of size $N - 1$ is equally likely to be a result of this swap. However,
through brute force calculation of derangements from size $5$ to $4$, it was revealed to us that
not all derangements of size $4$ are equally likely to be a result of a $1$-swap. This is a 
problem for our recurrence, as our recurrence assumes the derangements we start with are
uniformly distributed.

\subsection{swaps within derangements of size n lead to stable distribution - degree argument}
As highlighted in the previous section, when performing a $1$-swap and transitioning to a 
derangement of size $N - 1$, it is not true that all derangements of size $N - 1$ are
equally likely to occur, even though the recurrence assumes this property. This is
also the case for $N - 2$. Even then, through simulations we see that the average
number of swaps is consistent with our recurrence: $\frac{N^2}{4}$. In this section, we provide an argument for this.
In comes from the fact that the probability of a $0$-swap is much higher than $1$-swap or $2$-swap, thus
many $0$-swaps will be performed before eventually freezing something. We want to show that even if we start
off with a non uniform distribution of derangements size $N$, after performing $0$-swaps the distribution
will eventually stabilize into uniform. To see this, imagine a graph where the nodes are derangements
size $N$ and there exists an edge between two derangements if there is a $0$-swap from one to another.
Then we look at the degree of a derangement (aka the number of $0$-swaps in that particular derangement).
How can we find the degree of a derangement?

Let $D$ be a derangement of length $n$ with cycle decomposition $c_1, c_2, ..., c_k$.
Let $l_2$ be the number of cycles with length $2$.
We know the number of $0$-swaps in the derangement (or the degree) is

\[
    \sum_{i}{({c_i \choose 2} - c_i)} + \sum_{i, j}{{c_i}{c_j}} + l_2 
\]
\[
    = \sum_{i}{c_i \choose 2} + \sum_{i, j}{{c_i}{c_j}} - n + l_2 
\]
Note that we add $l_2$ because if $c_i$ is length 2 then 
$ {c_i \choose 2} - c_i = -1 $ thus we add $l_2$ to "ignore" the cycles of length 2.

I argue that the degree is ${n \choose 2} - n + l_2$. That is,

\[
    \sum_{i}{c_i \choose 2} + \sum_{i, j}{{c_i}{c_j}} - n + l_2 = {n \choose 2} - n  + l_2
\]
or equivalently,
\[
    \sum_{i}{c_i \choose 2} + \sum_{i, j}{{c_i}{c_j}} = {n \choose 2}
\]

\begin{proof}
        \begin{align}
            & \sum_{i}{c_i \choose 2} + \sum_{i, j}{{c_i}{c_j}}  \\\nonumber
            &=  \sum_{i}{c_i \choose 2} + \frac{(\sum_{i}{c_i})^2 - \sum_{i}{{c_i}^2}}{2} \\\nonumber
            &=  \frac{\sum_{i}{c_i(c_i -1)} + (\sum_{i}{c_i})^2 - \sum_{i}{{c_i}^2}}{2} \\\nonumber
            &=  \frac{n^2 + \sum_{i}{c_i(c_i - 1)} - \sum_{i}{{c_i}^2}}{2} \\\nonumber
            &=  \frac{n^2 - \sum_{i}{c_i}}{2} \\\nonumber
            &=  \frac{n^2 - n}{2} \\\nonumber
            &=  {n \choose 2}
        \end{align}
\end{proof}

Now, we see that the degree, also known as the number of 0 swaps in a derangement, is ${n \choose 2} - n + l_2 $.
Observe that the degree is dominated by the ${n \choose 2}$ term, so for large $n$, the degree of all of the
derangements are around the same. If the probability of a derangement corresponds to the degree, then this is a stable distribution.

\subsection{Transitioning to smaller derangements}
In the previous section, we study the action of performing a $0$-swap, aka going from derangement
size $n$ to derangement size $n$. In this section, we introduce some findings of going from
derangement size $n$ to $n - 1$ and $n - 2$. If we can prove that a $1$-swap
results in a uniform distribution of derangement size $n - 1$ (and similarly for $2$-swap), then we have
an argument for why the recurrence still holds: even if we start with a nonuniform distribution of
derangements size $n$, after performing enough $0$-swaps the distribution will stabilize. Then,
if we can argue the previous statement, then performing a $1$-swap will give us a uniform distribution
of derangements size $n - 1$.

Imagine a bipartite graph where one set of vertices is the derangements size $n$ and the
other is the set of derangements size $n - 2$. There is an edge from a derangement size
$n$ to $n - 2$ if there is a $2$-swap that results in that smaller derangement. 
In particular, a transition from a derangement size $n$ to $n - 2$ is a $2$-swap followed
by a renaming of rest $n - 2$ elements.
Observe that the outdegree of a derangement size $n$ is exactly the number of $2$ cycles ($l_2$).
So the outdegrees of the derangement size $n$ varies. But, we find that the indegrees of the
derangement size $n - 2$ is the same, which is ${n \choose 2}$. For the case
of derangements size $n - 1$, the indegrees are also the same, which is $n(n-1)$.

We can argue this by considering the reverse: how can we get a derangement size $n$ starting
with a derangement size $n - 1$? Let's take for example $n - 1 = 3$, so an example
derangement is $[3, 1, 2]$. To get a derangement size $n$, we choose one of the $n$ 
indices to be frozen. Let say we choose index $2$. So we know the position of one element:
$[\_, \_, 2, \_]$. We rename the vertices to get $[4, 1, 2, 3]$. Now we have a permutation
of size $n$, but to make it a derangement we must swap index $2$ with any of the other $n - 1$ indices.
So we had $n$ choices to pick an index to be frozen, and given that, we have $n - 1$ swaps that would
result in a derangement. So we have $n(n - 1)$. Given this argument, we can easily see why 
we get ${n \choose 2}$ in the case of derangements size $n - 2$. We pick two indices to be frozen
and only one swap would make it a derangement (swapping the two frozen elements) and we have ${n \choose 2}$ pairs.

\section{Adversary}
\subsection{describing the adversary}
Consider an adversary that tries to slow a blind sorting algorithm by delaying, as much as possible, any
element from freezing.
That is, if an algorithm performs a swap $(i, j)$, the adversary will try to claim that $A[i] \neq j$ and
$A[j] \neq i$. Of course, with enough swaps the adversary can no longer claim that an element doesn't
belong in a certain index. For example, if an element has visited $N - 1$ indices and still hasn't been frozen,
then we know it must belong in that $1$ other index.

With that, let's define a bipartite graph that
the adversary will maintain $G(V, E)$ where
the vertex set $V$ is split into two sets $L = {l_1, l_2, ..., l_n}$ and $I = {i_1, i_2, ..., i_n}$.
The set $L$ represents the elements of the array and the set $I$ represents the indices of the array.
We say an edge exists from vertex $l$ to vertex $i$ if the element $l$ can possibly be in index $i$.

When an algorithm performs a swap $(i, j)$ the adversary will say element $l = A[i]$ can not be in index $j$.
Thus, the edge going from $l$ to $j$ will be removed. The analagous edge for element $A[j]$ will also be removed.

In this bipartite graph, there must be a (maximum) matching at all times between the set of elements and the set
of indices. This matching represents a permutation of the array that is consistent with the swaps
performed so far. When a swap is requested, before removing the edge,
we must determine if after removing the edge if there is still a possible matching. If there isn't, then the adversary
can not claim that the element doesn't belong in this position. Therefore, it must report to the algorithm that
the element is frozen. In the next section, we will see how to implement this idea.

\subsection{Implementation}
The implementation of the adversary is as follows: we represent the graph as an adjacency list.
Except, instead of having a vertex map to a list we have it mapped to a hash set. This allows
for expected constant time deletion and insertion of edges.

As mentioned before, there is an directed edge $(l, i)$ if element $l$ can possibly belong to index $i$. 
If the edge is reversed (i.e $(i, l)$, then that signifies that edge is in our matching.
Initially, we start with element $l$ being matched to index $l+1$ for $l < n$ and index $1$ for $l = n$.

To reverse an edge $(u, v)$ we simply remove $v$ from $u$'s set and add $u$ to $v$'s set.

When we perform a swap, we attempt to remove the appropriate edges (there are two). Let's say
we are attempting to remove the edge $(u, v)$. We first remove $v$ from $u$'s vertex set. 
Then we use depth first search to see if there is an alternating path from $u$ to $v$ still.
If there is, then we update the matching by reversing the appropriate edges.
If there isn't a matching, then we can't remove the edge, and we tell the algorithm that
an element has been frozen.

\begin{minted}{python}
from collections import defaultdict

class Adversary:
    def __init__(self, n):
        # no derangements for n < 2
        assert not n < 2

        # build graph
        self.elements = ['element{}'.format(i) for i in range(n)]
        self.graph = defaultdict(set)

        for i, element in enumerate(self.elements):
            for j in range(n):
                if i != j:
                    self.graph[element].add(j)

                if (i + 1) % n == j:
                    self._reverse_edge(element, j)

    def _reverse_edge(self, u, v):
        self.graph[u].remove(v)
        self.graph[v].add(u)

    def _is_in_matching(self, element, idx):
        return self.graph[idx] == {element}

    def _exist_path(self, u, v):
        seen = set()
        path = []

        def dfs(node):
            if node == v:
                return True

            if node not in seen:
                seen.add(node)
                for adj in self.graph[node]:
                    path.append((node, adj))
                    found = dfs(adj)
                    if found:
                        return True
                    path.pop()

            return False

        return dfs(u), path

    def _attempt_remove(self, idx, element):
        # if edge is not in matching, simply remove
        if not self._is_in_matching(element, idx):
            if idx in self.graph[element]:
                self.graph[element].remove(idx)
            return True

        # try to remove, and see if there is an alternating path
        self.graph[idx].remove(element)
        found, path = self._exist_path(element, idx)
        if found:
            for u, v in path:
                self._reverse_edge(u, v)

            return True

        # no alternating path found, edge cannot be removed.
        # add edge back into graph
        self.graph[idx].add(element)
        return False

    def swap(self, i, j):
        # perform swap
        self.elements[i], self.elements[j] = self.elements[j], self.elements[i]

        # initialize return value
        frozen = []

        # can we say elements[i] cannot be in position i
        removed = self._attempt_remove(i, self.elements[i])
        if not removed:
            # if can't remove, add it to frozen
            frozen.append(i)

        # repeat for j
        removed = self._attempt_remove(j, self.elements[j])
        if not removed:
            frozen.append(j)

        return frozen

    def original_array(self):
        n = len(self.elements)
        arr = [None] * n
        for i in range(n):
            element = next(iter(self.graph[i]))
            element_number = int(element.strip('element'))
            arr[element_number] = i

        return arr
\end{minted}

Implementation of the deterministic algorithm that plays against the adversary
\begin{minted}{python}
from adversary import Adversary

def deterministic(n):
    A = Adversary(n)
    frozen = set()
    swaps = 0

    for i in range(n):
        j = i + 1
        while i not in frozen:
            if j not in frozen:
                frozen.update(A.swap(i, j))
                swaps += 1
            j += 1

    print(A.original_array())
    return swaps
\end{minted}

When the determinstic algorithm plays against the adversary for $N = 10$, the result of 
\begin{verbatim}
    print(A.original_array())
\end{verbatim}
is
\begin{verbatim}
    [6, 7, 8, 9, 10, 5, 4, 3, 2, 1]
\end{verbatim}
which is consistent with the worst case input we attempted to build earlier (along with the number of swaps).

\subsection{$n^2$ lower bound}
All elements are frozen when the matching in our graph is unique. That is, if we remove an edge
in the matching from the graph, then no other matching can be found. Let's consider two edges in the final
matching $(u, v)$ and $(k, l)$. We know that at least the edge $(u, l)$ or the edge $(v, k)$ must have been removed
from the graph, otherwise the matching is not unique. That is, for any pair of two edges (we have ${N \choose 2} - N$ pairs, since initially we are missing $N$ edges)
, we know at least $1$ edge has been removed. Since a swap removes at most $2$ edges, then at least $\frac{{N \choose 2} - N}{2}$ swaps have been performed,
thus establishing a lower bound of $\Omega(N^2)$ swaps for any Blind sorting algorithm.

\end{document}
