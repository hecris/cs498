\documentclass{article}

\usepackage{amsmath}
\usepackage{amssymb}
\usepackage{amsthm}
\newtheorem{lemma}{Lemma}
\newtheorem{conjecture}{Conjecture}
\newtheorem{theorem}{Theorem}
\newtheorem{proposition}{Proposition}
\newtheorem{corollary}{Corollary}
\newtheorem{definition}{Definition}

\title{Blind selection sort analysis}
\author{Christopher He}
\date{\today}

\begin{document}
\maketitle
\section{Exact running time}

Let $a$ be a permutation of $1,2, ..., N$.
Let $inv(a) = \{(i, j) \mid 1 \leq i < j \leq N, a_i > a_j \}$ be the set of inversions of $a$.
Let $ginv(a) = \{(i, j) \in inv(a) \mid a_j > j\}$ be the set of good inversions of a.
Let $inv_k(a) = \{(i, j) \in inv(a) \mid i = k\}$ denote inversions where the left is $k$,
and similarly define $ginv_k(a)$.

First we proved that if $(i, j) \in inv(a)$ and $a_j$ starts before its correct position $j$, then
so does $a_i$.

\begin{lemma}
    If $(i, j) \in ginv(a)$ then $a_i > i$.
    \begin{proof}
        If $(i, j) \in ginv(a)$ then $a_i > a_j > j > i$, so $a_i > i$.
    \end{proof}
\end{lemma}

We then proved that if an element starts before its position i.e $a_i > i$, the distance $a_i - i$ is 
at most $\vert inv_i(a) \vert$.

\begin{lemma} If $a_i > i$ then $a_i - i \leq \vert inv_i(a) \vert$.
    \begin{proof} There are $a_i - 1$ elements smaller than $a_i$, but only $i - 1$ available positions to its left. That means that at least $a_i - 1 - (i - 1) = a_i - i $ are on its right, forming that many inversions. So $ \vert inv_i(a) \vert \geq a_i - i$.
    \end{proof}
\end{lemma}

\begin{conjecture}
    The number of swaps $s(a)$ performed by blind selection sort on permutation $a$ is
    \begin{equation}
        s(a) = \sum_{i < a_i} a_i - i + \sum_{i > a_i} \vert inv_i(a) \vert - \sum_{i} \vert ginv_i(a) \vert
    \end{equation}
\end{conjecture}
\begin{proof}
    In general, the number of swaps involving $a_i$ (where $a_i$ is the left element) is equal to $ \vert inv_i(a) \vert $.
For example, let $a = [6, 5, 4, 3, 2, 1]$. The number of swaps involving $3$ is $2$.
When the algorithm searches for $1$, it will generate one swap for $3$. Another one will be generated
when $2$ is being searched.

    But notice that a swap with $a_i$ has a net effect
of shifting $a_i$ to the right. If an element starts before its position ($a_i > i$), then by Lemma 2, the element shifts
at most $a_i - i$ times before it is frozen.

    Lastly, we subtract $ \sum_{i} \vert ginv_i(a) \vert$ to account
    for that fact that elements may be frozen
    "accidentally", saving one swap for every unfrozen element before it. Note that by Lemma 1,
    $ \sum_{i} \vert ginv_i(a) \vert = \sum_{i < a_i} \vert ginv_i(a) \vert $.
\end{proof}

\begin{conjecture}
    The number of swaps $s(a)$ performed by blind selection sort on permutation $a$ is
    \begin{equation}
        s(a) = \sum_{i} inv_i(a) - 2 \sum_{i} \vert ginv_i(a) \vert
    \end{equation}
\end{conjecture}
\begin{proof}
    From Conjecture  1 we know the number of swaps is
    \begin{equation}
        s(a) = \sum_{i < a_i} a_i - i + \sum_{i > a_i} \vert inv_i(a) \vert - \sum_{i} \vert ginv_i(a) \vert
    \end{equation}
    If the following equality is proven (TODO: prove this by proving that good inversions balance the distance)
    \begin{equation}
        \sum_{i < a_i} a_i - i = \sum_{i \leq a_i} inv_i(a) - \sum_{i} ginv_i(a)
    \end{equation}
    then this statement is proven.
    \begin{align}
        &s(a) = \sum_{i < a_i} a_i - i + \sum_{i > a_i} \vert inv_i(a) \vert - \sum_{i} \vert ginv_i(a) \vert \\\nonumber
        &= \sum_{i \leq a_i} inv_i(a) - \sum_{i} \vert ginv_i(a) \vert + \sum_{i > a_i} \vert inv_i(a) \vert - \sum_{i} \vert ginv_i(a) \vert \\\nonumber
        &= \sum_{i} inv_i(a) - 2 \sum_{i} \vert ginv_i(a) \vert
    \end{align}
\end{proof}

\section{Structure of the worst case input}
\begin{theorem}
    If $a$ is the worst case permutation of $1, 2, ..., N$ for blind selection sort, then $ginv_i(a) = \emptyset$.
    \begin{proof}
        Assume $ginv_i(a) \neq \emptyset$, then pick $(i, j) \in ginv_i(a)$ such that $\forall k, i < k < j, a_k < k$.
        So we start with $a = [a_1, ... , a_i, ...,  a_j, ..., a_N]$. We can construct a new permutation by swapping $a_i$ and $a_j$
        i.e $a' = [a_1, ... , a_j, ...,  a_i, ..., a_N]$. This effectively removes the good inversion $(i, j)$.
        Furthermore, let $k, i < k < j$. The number of inversions involving $k$ remains the same after the swap. TODO: finish.
        By Conjecture 2, $s(a') = s(a) + 1$, thus contradicting that $a$ is the worst case.
    \end{proof}
\end{theorem}
This gives us the immediate result that in the worst input $a$, if an element starts before its position, all elements before it are smaller (otherwise $ginv(a) \neq \emptyset$).
\begin{corollary}
    In the worst case permutation of $1, 2, ... , N$ for blind selection sort, if $a_j > j$ then $\forall i < j, a_i < a_j$.
\end{corollary}
\begin{theorem}
    The worst case permutation of $1, 2, ..., N$ for blind selection sort starts with an increasing, consecutive sequence up to $N$.
    \begin{lemma}
        If $a$ is the worst input for blind selection sort and $a_j > j$, then $ \forall i < j, a_i > i$.
        \begin{proof}
            We will prove the equivalent statement: if $j > 1, a_j > j$, then $a_{j - 1} > j - 1$.
            Assume for the sake of contradition
            that $a_{j - 1} < j - 1$. By Corollary 1, we know $a_{j - 1} < a_j$. So switch the position of
            $a_{j - 1}$, and $a_j$. We created an inversion, and we know this is not a good inversion since
            $a_{j - 1} < j - 1 < j $  (in other words, after moving $a_{j - 1}$ to the right, it is still after
            its correct position.) So by Conjecture 2, we made an even worse input.
        \end{proof}
    \end{lemma}
    \begin{definition}
        If $a_k = x$ then  $idx(x) = k$.
    \end{definition}
    \begin{lemma}
        If $a$ is the worst input for blind selection sort and $\exists j > 1$ such that $a_j > j$, then $ a_{j-1} = j - 1$.
        \begin{proof}
            Assume for the sake of contradiction that $a_{j - 1} \neq j - 1$ and
            let $k = idx(j - 1)$. We know that $k > j - 1$, otherwise $(k, j - 1) \in ginv(a)$. Switch
            $a_{j - 1}$ with $a_k$.
        \end{proof}
    \end{lemma}
    \begin{proof}
        Since $a_{N} < N$ (otherwise it is frozen), then by Lemma 2 and 3, the statement is proven.
    \end{proof}
\end{theorem}

\end{document}

